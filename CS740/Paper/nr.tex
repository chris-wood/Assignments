\documentclass[10pt,a4paper,onecolumn]{article}
\usepackage[latin1]{inputenc}
\usepackage{amsmath}
\usepackage{amsfonts}
\usepackage{amssymb}
\usepackage{makeidx}
\usepackage{graphicx}
\author{Christopher A. Wood}
\title{Trends in Network Robustness Research}
\date{\today}
\begin{document}
\maketitle

\begin{abstract}
Due to the growing pervasiveness of civilian and 
military networks for the transmission of safety-critical and real-time data, 
it is critically important that they are resistant to selective and random network 
node deletions. Network robustness is a measure of the performance and 
throughput responsiveness of a network in response to such deletions. The nature of 
this metrics lends itself to the application of percolation theory, which can be
used to describe the behavior of connected clusters in a random graph. This theory
can be utilized to design and construct optimally robust networks in order to yield
the best performance in the event of node deletions.

This paper presents some background information on network robustness and
its importance in modern communication systems, presents some recent advances
made in the topic, and concludes with avenues of future work that can be explored
by researchers in the field. 
\end{abstract}

% http://crpit.com/confpapers/CRPITV26Dekker.pdf
% http://en.wikipedia.org/wiki/Percolation_theory
% http://arxiv.org/abs/cond-mat/0007300
% http://www.research.ibm.com/people/r/rish/papers/PhysicaA_3_30.pdf
% http://www.cs.berkeley.edu/~dawnsong/papers/coloring_ndss08_cr.pdf

% robust routing and dynamic load balancing - the hw/sw solution to help deal with robustness and traffic changes
% http://iie.fing.edu.uy/investigacion/grupos/artes/publicaciones/casas_drcn09.pdf

\section{Introduction}
TODO

\end{document}