\documentclass[10pt,a4paper,onecolumn]{article}
\usepackage[latin1]{inputenc}
\usepackage{amsmath}
\usepackage{amsfonts}
\usepackage{amssymb}
\usepackage{makeidx}
\usepackage{graphicx}
\author{Christopher A. Wood}
\title{Trends in Network Robustness Research}
\date{\today}
\begin{document}
\maketitle

\begin{abstract}
Network robustness is a measure of the integrity of a network in the event that
certain nodes are removed from the same network. It is a fundamental topic in
percolation theory, which is used to describe the behavior of connected clusters
in a random graph. Following the growing usage and importance of civilian and 
military networks, it is important that these networks are robust so as to still 
yield high performance in the event that a single node drops out of the network.
This paper presents some background information on network robustness and
its importance in modern communication systems, presents some recent advances
made in the field, and concludes with avenues of future work that can be explored
by researchers in the field. 
\end{abstract}

% http://crpit.com/confpapers/CRPITV26Dekker.pdf
% http://en.wikipedia.org/wiki/Percolation_theory
% http://arxiv.org/abs/cond-mat/0007300

\end{document}