\documentclass[11pt]{article}

\usepackage{thumbpdf, amssymb, amsmath, amsthm, microtype,
	    graphicx, verbatim, listings, color, fancybox}
\usepackage[pdftex]{hyperref}
%\usepackage[margin=1in]{geometry}
\usepackage{cawsty}
\usepackage{fullpage}
\usepackage{pseudocode}

%\setlength{\parindent}{0pt}

\linespread{1.2}

\begin{document}
\arltitle{4040-849 Optimization Methods}{Written Assignment 2}
\begin{prob}{1-a}
\end{prob}
\begin{sol} 

Making the substitution of $f(\lambda)$ for $\frac{\tau_{zy}}{p_{max}}$, where $\lambda = \frac{z}{b}$, we get a simplified equation that can be simplified as follows.

\begin{eqnarray*}
f(\lambda) & = & -\frac{1}{2}\Bigg[ -\frac{1}{\sqrt[]{1 + \lambda^2}} +  \Bigg(2 - \frac{1}{1 + \lambda^2}\Bigg) \sqrt[]{1 + \lambda^2} -2\lambda \Bigg] \\
& = & -\frac{1}{2}\Bigg[ -\frac{1}{\sqrt[]{1 + \lambda^2}} +  2\sqrt[]{1 + \lambda^2} - \frac{\sqrt[]{1 + \lambda^2}}{1 + \lambda^2} -2\lambda \Bigg] \\
& = & \frac{0.5}{\sqrt[]{1 + \lambda^2}} - \sqrt[]{1 + \lambda^2} + \frac{0.5\sqrt[]{1 + \lambda^2}}{1 + \lambda^2} + \lambda \\
& = & \frac{0.5}{\sqrt[]{1 + \lambda^2}} - \sqrt[]{1 + \lambda^2}\Bigg(1 - \frac{0.5}{1 + \lambda^2}\Bigg) + \lambda
\end{eqnarray*}

Therefore, as shown, we can reduce the problem of finding the location of the maximum shear stress for $v_{1} = v_{2} = 3$ reduces to maximizing the function shown below:

\begin{eqnarray}
\label{OriginalFunc}
f(\lambda) = \frac{0.5}{\sqrt[]{1 + \lambda^2}} - \sqrt[]{1 + \lambda^2}\Bigg(1 - \frac{0.5}{1 + \lambda^2}\Bigg) + \lambda
\end{eqnarray}

\end{sol}

\begin{prob}{1-b}
\end{prob}
\begin{sol} 

In order to apply the Fibonacci method, we must be trying to minimize the objective function for a particular problem. Therefore, since we were given an objective function (\ref{OriginalFunc}) that we must maximize, we simply negate this function so that we can apply the Fibonacci method to solve it numerically. The resulting function that we seek to minimize is shown below.

\begin{eqnarray*}
f'(x) = -f(x) =-\frac{0.5}{\sqrt{1+x^2}}+\sqrt{1+x^2} \left(1-\frac{0.5}{1+x^2}\right)-x 
\end{eqnarray*}

The values for $J, A_{1}, B_{1}, L_{1}, L_{2}^*$ from each iteration of the Fibonacci method from $J=2$ to $J=8$ are shown below, starting with initial values of $0$ and $3$ for $A_{1}$ and $B_{1}$, respectively.

\begin{center}
  \begin{tabular}{| c | c | c | c | c |}
    \hline
	J & $A_{1}$ & $B_{1}$ & \textbf{$L_{1}$} & \textbf{$L_{2}^{*}$} \\ \hline
	2 & 0 & 3 & 3 & $\frac{39}{34}$ \\ \hline
	3 & 0 & $\frac{63}{34}$ & $\frac{63}{34}$ & $\frac{39}{34}$ \\ \hline
	4 & 0 & $\frac{39}{34}$ & $\frac{39}{34}$ & $\frac{12}{17}$ \\ \hline
	5 & $\frac{15}{34}$ & $\frac{39}{34}$ & $\frac{12}{17}$ & $\frac{15}{34}$ \\ \hline
	6 & $\frac{15}{34}$ & $\frac{15}{17}$ & $\frac{15}{34}$ & $\frac{9}{34}$ \\ \hline
	7 & $\frac{21}{34}$ & $\frac{15}{17}$ & $\frac{9}{34}$ & $\frac{3}{17}$ \\ \hline
	8 & $\frac{12}{17}$ & $\frac{15}{17}$ 
  \end{tabular}
\end{center}

\end{sol}

\begin{prob}{1-c}
\end{prob}
\begin{sol} 

In order to apply Newton's method to find a critical point of the given objective function, we must first derive the first derviative of $f(\lambda)$ with respect to $\lambda$. The original function and its derivative are shown below.

\begin{eqnarray*}
f(\lambda)=\frac{0.5}{\sqrt{1+\lambda^2}}-\sqrt{1+\lambda^2} \left(1-\frac{0.5}{1+\lambda^2}\right)+\lambda
\end{eqnarray*}
\begin{eqnarray*}
f'(\lambda)=\frac{\lambda \left(-\lambda^2-2.\right)}{\left(\lambda^2+1\right)^{3/2}}+1 \\
\end{eqnarray*}
%\begin{eqnarray*}
%f''(\lambda)=\frac{\lambda^2-2.}{\sqrt{\lambda^2+1} \left(\lambda^2+1.\right)^2}
%\end{eqnarray*}

Iteratively applying the Newton method with $\lambda = 0.6$ to start, we end up with the following results:

\begin{center}
  \begin{tabular}{| c | c | c | c | c |}
    \hline
	$\lambda_{n}$ & $f(\lambda)$ & $f'(\lambda)$ & $\lambda_{n+1}$ & $\epsilon$ \\ \hline
	0.6 & 0.291303 & 0.107199 & -2.11741 & 2.06913 \\ \hline
	
  \end{tabular}
\end{center}

\end{sol}

\begin{prob}{1-d}
\end{prob}
\begin{sol} 



\end{sol}

\end{document}
