\documentclass[11pt]{article}

\usepackage{thumbpdf, amssymb, amsmath, amsthm, microtype,
	    graphicx, verbatim, listings, color, fancybox}
\usepackage[pdftex]{hyperref}
%\usepackage[margin=1in]{geometry}
\usepackage{cawsty}
\usepackage{fullpage}
\usepackage{pseudocode}

%\setlength{\parindent}{0pt}

\linespread{1.2}

\begin{document}
\arltitle{4040-849 Optimization Methods}{Written Assignment 1}
\begin{prob}{1}
\end{prob}
\begin{sol} 

The information provided about the mixed nuts, their content percentage, and cost can be displayed in tabular form as follows: \\

\begin{center}
\begin{tabular}{|l|l|l|}
\hline
\textbf{Nut} & \textbf{Type A (\%)} & \textbf{Type B (\%)} \\
\hline
Almonds & 0.20 & 0.10 \\
Cashew nuts & 0.10 & 0.20 \\
Walnuts & 0.15 & 0.25 \\
Peanuts & 0.55 & 0.45 \\
\hline
\textbf{Cost} & \$2.50 & \$3.00 \\
\hline
\end{tabular}  \\
\end{center}

From this information we can determine the following properties of the problem: \\

\textbf{Design variables:}\\
\begin{eqnarray*}
x = \text{ the number of pounds of mixed nut type} A \\
y = \text{ the number of pounds of mixed nut type} B
\end{eqnarray*}

\textbf{Constraints:} \\
\begin{eqnarray*}
0.2x + 0.1y \geq 4 \\
0.1x + 0.2y \geq 5 \\
0.15x + 0.25y \geq 6
\end{eqnarray*}

\textbf{Cost function:}
\begin{eqnarray*}
f(x,y) = 2.5x + 3y
\end{eqnarray*}

TODO: convert into standard form now... (equality with constraints, etc..)
\end{sol}

\begin{prob}{2}
\end{prob}
\begin{sol} 

Based on the problem description, we can identify the following properties of the problem: \\

\textbf{Design variables:}\\ 
%\begin{eqnarray*}
$x_{1,1}$ = \#resources between New York to Seattle\\
$x_{1,2}$ = \#resources between New York to Houston\\
$x_{1,3}$ = \#resources between New York to Detroit \\
$x_{2,1}$ = \#resources between Los Angeles to Seattle \\
$x_{2,2}$ = \#resources between Los Angeles to Houston \\
$x_{2,3}$ = \#resources between Los Angeles to Detroit \\
%\end{eqnarray*}

\textbf{Constraints:} 
\begin{eqnarray*}
x_{1,1} + x_{1,2} + x_{1,3} \leq 3 \\
x_{2,1} + x_{2,2} + x_{2,3} \leq 3 \\
x_{1,1} + x_{2,1} = 2 \\
x_{1,2} + x_{2,2} = 3 \\
x_{1,3} + x_{2,3} = 1 
\end{eqnarray*}

\textbf{Cost function:}
\begin{eqnarray*}
f(x_{1,1},x_{1,2},x_{1,3},x_{2,1},x_{2,2},x_{2,3}) = 4x_{1,1} + 3x_{1,2} + x_{1,3} + 2x_{2,1} + 7x_{2,2} + 5x_{2,3}
\end{eqnarray*}

TODO: convert into standard form now... (equality with constraints, etc..)

\end{sol}

\begin{prob}{3}
\end{prob}
\begin{sol} 
This problem can be classified as an optimal control problem with 9 total states (including an initial state where no money has been invested yet). Thus, the problem can be formulated with the following parameters:\\

\textbf{Design variables:}\\
%\begin{eqnarray*}
$i$ = current state of the problem (state) \\
$X_{i,j}$ = amount of money invested in CD type $j$ at state $i$ (control) \\
$y_{i,a}$ = amount of money available for investment at state $i$ (state) \\
$y_{i,c}$ = amount of money accumulated thus far at state $i$ (state) \\
%\end{eqnarray*}

\textbf{Constraints:} 
\begin{eqnarray*}
y_{0,a} = 50000 \\
y_{i,a} \geq 0 \\
x_{i,1} + x_{i,2} + x_{i,3} + x_{1,4} \leq y_{i,a}
\end{eqnarray*}

\textbf{Cost function:}
\begin{eqnarray*}
f(X) = \sum_{i=0}^{8} y_{i,c} = 0.05X_{i-1,1} + 0.07X_{i-2,2} + 0.1X_{i-4,3} + 0.15X_{i-8,4}
\end{eqnarray*}

\end{sol}

\end{document}
