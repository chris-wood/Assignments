\documentclass[11pt]{article}

\usepackage{thumbpdf, amssymb, amsmath, amsthm, microtype,
	    graphicx, verbatim, listings, color, fancybox}
\usepackage[pdftex]{hyperref}
%\usepackage[margin=1in]{geometry}
\usepackage{cawsty}
\usepackage{fullpage}
\usepackage{pseudocode}

\newcommand{\field}[1]{\mathbb{#1}} %requires amsfonts

%\setlength{\parindent}{0pt}

\linespread{1.2}

\begin{document}
\cawtitlelong{4040-849 Optimization Methods}{Optimizing Cryptographic Strength of Substitution}{Layers in Symmetric-Key Cryptosystems}

\begin{abstract}
The cryptographic security of symmetric-key block ciphers and other related primitives is based upon their adherence to Shannon's principles of confusion and diffusion \cite{Kim90astudy}. Confusion can be defined as the statistical relationship between the ciphertext and private key of a cipher, while diffusion refers to the statistical redundancy of plaintext bits in the ciphertext bits. Consequently, it is increasingly important to optimize these characteristics in order to make them less susceptible to attacks based on linear and differential cryptanalysis. S(ubstitution)-boxes are the most traditional mathematical structures that are used to improve the levels of diffusion and confusion within symmetric-key cryptographic algorithms. Recent research efforts have revealed practical measurements of S-box constructions that indicate their susceptibility to linear and differential cryptanalysis. In this work, we attempt to formulate the problem of cryptographically strong substitution layers in symmetric-key block ciphers with S-box designs into a mixed integer programming problem that can be optimized to yield the high diffusion and confusion dividends in resulting cipher implementations.
\end{abstract}

\section{Problem Description}
%TODO: include information from report in here

\section{Optimization Solution}
%TODO: summarize genetic algorithm approach, state each optimization problem as a MINLP problem

\section{Optimization Analysis}
%TODO: compare GA results against exhaustive search and BB algorithm

\end{document}
