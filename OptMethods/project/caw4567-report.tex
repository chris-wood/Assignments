\documentclass[11pt]{article}

\usepackage{thumbpdf, amssymb, amsmath, amsthm, microtype,
	    graphicx, verbatim, listings, color, fancybox}
\usepackage[pdftex]{hyperref}
%\usepackage[margin=1in]{geometry}
\usepackage{cawsty}
\usepackage{fullpage}
\usepackage{pseudocode}

\newcommand{\field}[1]{\mathbb{#1}} %requires amsfonts

%\setlength{\parindent}{0pt}

\linespread{1.2}

\begin{document}
\arltitle{4040-849 Optimization Methods}{Optimizing Cryptographic Boolean Function Constructions}

%TODO: define boolean function
\begin{abstract}
Cryptographically secure block ciphers are based around Shannon's principles of confusion and diffusion \cite{Kim90astudy}. It is important to optimize these characteristics in order to make ciphers less susceptible to linear and differential cryptanalysis attacks. The most traditional way to integrate mathematical structures that improve the confusion of a block cipher is to use a substitution box  (or simply, an S-box). Recent research efforts have revealed practical measurements of S-box constructions that indicate their susceptibility to linear and differential cryptanalysis. In this work, we attempt to formulate the problem of cryptographically strong S-box designs (and subsequently, any confusion layer design) into an integer programming problem that can be optimized to yield the highest confusion dividends in resulting cipher implementations.
\end{abstract}

\section{Problem Description and Background Information}

Mathematically, an S-box can be represented as a function $f$ that maps input values $a$ to output values $b$ such that $a,b \in \field{F}_2^n$. In cryptographic terms, such a function $f$ must be bijective in order to avoid bias towards any specific output element in the field. We now present a series of definitions that are pertinent to the design of cryptographically strong S-Boxes (or confusion layers) \cite{Mar_newanalysis}.

%TODO: http://www.waset.org/journals/waset/v48/v48-24.pdf && thesis work

\begin{define}
The Hamming weight of an element $a \in \field{F}_2^n$ is defined as wt$(x) = \sum x_i$.
\end{define}

\begin{define}
Let $f$ be a bijective function with range $\mathbb{R^*}$, where $|\mathbb{R^*}| = m$. Let $n$ be the number of elements $x$ that satisfy $f(x \oplus \Delta_i) = f(x) \oplus \Delta_o$. Then, $\frac{n}{m}$ is the \emph{differential probability p} of the characteristic $f_D(\Delta_i \to \Delta_o)$.
\end{define}

\begin{define}
The \emph{branch number} of an $n \times n$-bit S-Box is
\begin{eqnarray*}
BN = \text{min}_{a, b\not=a}(\text{wt}(a \oplus b) + \text{wt}(S(a) \oplus S(b))),
\end{eqnarray*}
where $a, b \in \field{F}_2^n$.
\end{define}

\begin{define}
A function $f : \field{F}_2^n \to \field{F}_2^n$ exhibits the avalanche effect if and only if 
\begin{eqnarray*}
\sum_{x \in \field{F}_2^n} \text{wt}(f(x) \oplus f(x \oplus c_{i}^{n})) = n2^{n-1},
\end{eqnarray*}
for all $i (1 \leq i \leq n)$, where $c_{i}^{n} = [0, 0, ..., 1, ..., 0]$ (where a $1$ is in the $n$th position of the vector of cardinality $n$.
\end{define}

\begin{define}
A function $f : \field{F}_2^n \to \field{F}_2^n$ satisfies the Strong Avalanche Critertion (SAC) if for all $i (1 \leq i \leq n)$ the following equations hold:
\begin{eqnarray*}
\sum_{x \in \field{F}_2^n} f(x) \oplus f(x \oplus c_i^n) = (2^{n-1}, 2^{n-1}, ..., 2^{n-1})
\end{eqnarray*}
This simply means that the $f(x) \oplus f(x \oplus c_i^n)$ is balanced for every element in $\field{F}_2^n$ with Hamming distance of $1$. 
\end{define}

Ideal construction of cryptographic primitives will utilize internal boolean functions that satisfy the SAC criterion because they result in high levels of confusion, thus thwarting attempts by an attacker to statistically relate the ciphertext of a cipher to the key that was used for encryption or decryption. However, in order to prevent differential cryptanalysis attacks, it is important that these boolean functions also have a high branch number.

Strong S-Boxes also exhibit strong non-linearity properties \cite{Kim90astudy}. It has been shown by Rueppel that the nonlinearity of a boolean function can be measured by the Hamming distance to the set of affine transformations and is related to the Walsh transform $\hat{F}$ of $\hat{f} : \field{F}_2^n \to \{-1, 1\}$ according to:
\begin{eqnarray*}
\delta(f)  = 2^{n-1} - \frac{1}{2}\text{max}_w|\hat{F}(w)|,
\end{eqnarray*}
where $\hat{F}(w)$ is the Walsh transformation defined as follows:
\begin{eqnarray*}
W_f(a) = \sum_{x \in \field{F}_2^n} (-1)^{f(x) + <a,x>},
\end{eqnarray*}
where $<a,x>$ is the scalar product of $a$ and $x$ (if they are thought of as vectors).
%\begin{define}
%TODO: nonlinearity Walsh transform 
%\end{define}

Designers of cryptographically secure cryptographic primitives (e.g. block ciphers, hash functions, etc) use these measurements as a basis for their susceptibility to linear and differential cryptanalysis when constructing their algorithms. However, they place the additional constraint on such primitives that fast and simple mathematical operations must be used to emulate a bijective function $f$ that exhbits ideal properties for all of these values.

As previously mentioned, S-Boxes have been the primary source of confusion in most cryptographic primitives. However, some recently proposed cryptographic primitives (e.g. the Skein hash function \cite{Ferguson09theskein}) are making use of an ARX (Addition, Rotation, and XOR) layer to provide similar results. This design decision was driven by the lack of strict security proofs for S-Boxes and the implementation efficiency of the ARX design. However, at an abstract level, an ARX layer can be represented by a bijective function $f$ as well, thus making it applicable to this problem. However, because the security of ARX confusion layers is not well understood and has only recently faced considerable cryptanalysis research \cite{Khovratovich:2010:RCA:1876089.1876116}, we restrict ourselves to S-Box designs.

\section{Optimization Candidate Description}
Cryptographically secure primitives utilize confusion layers that offer the following characteristics:

\begin{enumerate}
	\item Low differential propagation probability
	\item High branch number
	\item Satisfaction of the SAC criterion
	\item High degree of non-linearity
\end{enumerate}

Therefore, it is natural to reduce the problem of finding an optimal confusion layer for cryptographic primitives to an integer programming problem that seeks to optimize each one of these construction dimensions. In other words, optimal confusion layer construction can be thought of an integer programming problem with multiple cost functions that share a single, balanced solution. The common solutions to these objectives (if they exist) are thus contained within the Pareto set for the problem.

In this work we will seek to abstract the construction of confusion layers away from the Boolean functions that they represent and optimize the representation of this function in order to achieve ideal values for aforementioned metrics. In other words, we focus on the construction of a function $f$ with finite domain and range (both of the same cardinality) that could potentially be realized by a mathematical operation or construction. Only the forward version of $f$ shall be considered in this construction. However, in practical settings, $f$ must be invertible to be applied in symmetric key cryptosystems. 

Therefore, the design variables pair-wise mappings of the bijective function $f$. In order to manage the complexity of the problem, only 4-bit layers will be considered (i.e. $|\mathbb{R^*}| = 2^4 = 16$).

\section{Programming Language Use}
The Optimization Toolkit will be utilized in conjunction with MatLab in order to perform the mathematical computations. In addition, Mathematica will be utilized to allow for easy data collection management and visualization during the course of the project. 

%TODO: include one more reputable source of information
\bibliographystyle{IEEEtran}
% argument is your BibTeX string definitions and bibliography database(s)
\bibliography{caw4567-report}

\begin{comment}
Key distribution is a vital part of wireless ad-hoc communication networks that need to transfer text, audio, and video data both securely and efficiently. Traditionally, key agreement protocols are based upon the commonly known Diffie Hellman key exchange protocol, in which two (or more) parties may exchange public information in order to establish a common key. The problem with this approach is that it is very computationally inefficient, and doesn't lend itself directly to the problem of establishing a single common group key among multiple parties in a group. This is especially true in wireless ad-hoc networks where the nodes themselves have constrained processing and power resources.

For this reason, key agreement schemes for this specific type of network typically rely on pre-placed information that can be easily distributed to members of the group in order to establish a common group key. Therefore, at the physical layer of the network, where the group topology is represented as a single spanning tree, it is important that the latency of sending data between two nodes is as small as possible in order to ensure the fastest transmission of data. Depending on the radio propagation model and specific waveform used to propagate the digital data via an analog signal to each of the nodes, the structure of this tree can have a drastic impact on the time it takes to distribute a specific piece of data from node to every other node in the group.

Therefore, the purpose of this project is to minimize the time it takes this data to transmit to every other node in the group depending on the following parameters:

\begin{enumerate}
	\item Number of slots available in the TDMA scheme 
	\item Maximum number of nodes in the network
	\item Maximum number of node children allowed in the spanning tree
	\item Data packet size
	\item Radio channel bandwidth
\end{enumerate}

%Find minimum key distribution times for spanning tree representation of wireless ad-hoc networks.
% 1. Design variables:
% number of slots available in TDMA scheme (if TDMA is taken into account)
% number of nodes in the network
% maximum number of node children allowed
% key packet size
% channel bandwidth
% 2. Constraints
% sum of children
% all node children <= max number of children
% #used TDAM slots <= #available TDMA slots
% #
\end{comment}

\end{document}
