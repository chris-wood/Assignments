\documentclass[12pt,letterpaper]{article}
\usepackage[margin=1.0in]{geometry}

\begin{document}

\begin{center}
Christopher Wood \\
Reflective Essay \#3 \\
\end{center}

%Thesis:  \\
% benefit: can yield more abstract encodings of problem solutions (say examples below)
% can help quickly employ techniques from both paradigms to help devise and program solutions (leads to evolutionary development of software)

% CURRY HAS BEST OF BOTH WORLDS MAINLY BY
%1. benefit: nondeterminism joined with functional evaluation is extremely powerful
%	the example with regular expressions is very helpful, and a common problem that programmers face
%	computations with incomplete information
%2. benefit: search strategies for solving nondeterminism in equations increase declarative nature of programs
%	don't imperatively search for solutions, allow non-determinism 	
%	you can define rules in terms of functional patterns
% 	rely on rules of the program to determine functional evaluation with decision trees, not case expressions hand-coded by the programmer
%3. good for prototyping (exploring search space)
%	when algorithm is unknown, helps clarify problems, search for paths to solutions using constraints on the problem domain, achieved by completeness and soundness of search algorithm
%4. can only use features from the paradigm that are desired (don't have free variables and multiple rules if we want pure functionality, and don't use functions if we want pure logic)

% bad side for multiple paradigms: complexity of execution (example with search strategy), and available languages (we see features from both paradgims spilling over into 
%4. bad: search strategies are inefficient and aren't able to keep up with modern parallelism (functional languages don't have this issue because they are deterministic)
%	programmer needs to explicitly define rules in order to control search strategies (constructive guesses)
%	of course, demand-driven evaluation of functions leads to more efficiency
%	research is being done to make up for this, but no proof that it's more efficient yet

% conclusion: functional composition (func) and nondeterminism (logic) combined are good, technology is picking up speed to catch up, and we will hopefully see adoption of this language or major influences on other languages from this research


In recent years there has been increasing evidence of the integration of programming language paradigms.
Perhaps the most significant signs of this merging seen in industry are the presence of lambda 
operators in highly imperative languages like C\# and Python. Such features simplify many tasks in an
imperative language, such as searching and filtering data structures. However, while these efforts are the result of merging two separate domains, there has also been similar work done combining multiple 
declarative paradigms together. Functional logic programming, a dichotomy of separate and seemingly 
conflicting paradigms, has been shown to be quite useful for a variety of reasons. We present a
comparative analysis of the benefits and drawbacks of this new multi-language paradigm in the context of
Curry, the most prominent functional logic programming language available.

Perhaps the greatest benefit of multi-language paradigms is the ability to selectively apply helpful 
programming techniques from any supported language. For example, Curry gives programmers the ability to 
make use of nondeterminism with free variables and multiple rule declarations, while at the same time 
supporting functional composition to aid in algorithmic abstractions. The programmer is then left to 
make use of both of these features at their leisure, depending on the specific problem they may be 
trying to solve. Using them together can also prove to be very useful as well. Antoy and Hanus (cite) 
present an example of encoding a regular expression matcher in Curry that uses the nondeterminism through
multiple rules and functional patterns to determine if a given input string matches an ended regular 
expression. Fortunately, Curry's demand-driven search strategy limits the computational overhead of 
the incorporating free variables into function parameters. However, the programmer can also modify
the rules of the regular expression to place further constraints on the problem (such a technique is 
referred to as ``constrained construction''). Put another way, the benefit of having these different 
language features at your disposal is an attempt to make the language more declarative. 

In the context of Curry, another major benefit of its functional and logic juxtaposition is that 
is enables programmers to more effectively prototype their software solutions to particular problems.
If a solution or an efficient algorithm is not known, the nondeterministic nature of Curry, combined
with the functional representation of the problem, can be used to explore the steps that are used
to find a solution. This might lead to additional insight about the problem at hand and also inspire
programmers to devise creative algorithms to generate correct solutions. Speaking in generalities, 
this is effectively a benefit gained from the ability to encode problem solutions in more abstract 
descriptions.

Unfortunately, as with any level of software abstraction, there is a price to be paid for the added 
convenience factors that make software development easier. In Curry, the search strategy employed 
to intelligently search the state space of a free variable is motivated by demand-driven evaluations, 
constrained construction, and the use of defitional trees to structure re-write rules. Even with
these improvements, the search space of a free variable is still determined by the rule and equation
constraints imposed by the programmer. If these are not very well known and the programmer wants to 
explore the search space for a solution, then the search strategy cannot optimize its behavior to yield
acceptable performance. 

Furthermore, multi-language abstractions can also introduce conflicts with one of the languages, thus
yielding negative effects when programs are run. As an example, Curry search strategies are not able to 
keep up with processor-level parallelism. Strictly functional languages are very capable of making use
of multiple cores because they are deterministic. The nondeterminism in Curry hinders the execution 
from spreading across multiple cores because ***.

Overall, we see that there are indeed many benefits and drawbacks to multi-language paradigms. Antoy and 
Hanus have shown us many examples with their discussion of Curry. However, the biggest extrapolation 
from their article is that it is extremely important to cosnider all of the implications of a 
multi-language paradigm before readily adopting it for full-time use. There are many theoretical and 
practical advantages and disadvantages to such technologies, and as with any programming language, 
the target language is best chosen by examining its usefulness in the context of a specific problem at 
hand. There is no ``one size fits all'' language that any merging of languages can solve.


\bibliographystyle{plain}
\bibliography{ref}

\end{document}