\documentclass[11pt]{article}

\usepackage{thumbpdf, amssymb, amsmath, amsthm, microtype,
	    graphicx, verbatim, listings, color, fancybox}
\usepackage[pdftex]{hyperref}
%\usepackage[margin=1in]{geometry}
\usepackage{cawsty}
\usepackage{fullpage}
\usepackage{pseudocode}
\usepackage{verbatim}

\newcommand{\tlg}{\text{lg}}
\newcommand{\tln}{\text{ln}}
\newcommand{\tlog}{\text{log}}

\usepackage{algorithm}
%\usepackage{algorithmic}
\usepackage{amsmath}
\usepackage{amsthm}
\usepackage{algpseudocode}
\usepackage{algorithmicx}% http://ctan.org/pkg/algorithmicx
\usepackage{lipsum}% http://ctan.org/pkg/lipsum
\usepackage{xifthen}% http://ctan.org/pkg/xifthen
\usepackage{needspace}% http://ctan.org/pkg/needspace
\usepackage{hyperref}% http://ctan.org/pkg/hyperref

\usepackage{tikz}
\usetikzlibrary{arrows,%
                shapes,positioning}

\tikzstyle{vertex}=[circle,fill=black!25,minimum size=20pt,inner sep=0pt]
\tikzstyle{selected vertex} = [vertex, fill=red!24]
\tikzstyle{edge} = [draw,thick,-]
\tikzstyle{weight} = [font=\small]
\tikzstyle{selected edge} = [draw,line width=5pt,-,red!50]
\tikzstyle{ignored edge} = [draw,line width=5pt,-,black!20]

\allowdisplaybreaks[1]

% ================ ALGORITHM ENVIRONMENT ================
\newcounter{numberedAlg}% Algorithm counter
\newenvironment{numberedAlg}[1][]%
  {% \begin{numberedAlg}[#1]
    \needspace{2\baselineskip}% At least 2\baselineskip required, otherwise break
    \noindent \rule{\linewidth}{1pt} \endgraf% Top rule
    \refstepcounter{numberedAlg}% For correct reference of algorithm
    \centering \textsc{Algorithm}~\thenumberedAlg%
    \ifthenelse{\isempty{#1}}{}{:\ #1}% Typeset name (if provided)
  }{% \end{numberedAlg}
  \noindent \rule{\linewidth}{1pt}% Bottom rule
  }%

%\setlength{\parindent}{0pt}

\linespread{1.2}

\begin{document}
\cawtitle{4005-800 Algorithms}{Homework 4}
\begin{prob}{1-a}
Prove that for any $n \in \mathbb{N}, \lfloor(n+1)/2\rfloor = \lceil n/2 \rceil$.
\end{prob}
\begin{sol}

\textbf{Case 1:} $n$ is even ($2$ $|$ $n$, or $n = 2m$ for some $m \in \mathbb{N}$) 
\begin{eqnarray*}
\Big\lfloor\frac{n+1}{2}\Big\rfloor & = & \Big\lfloor\frac{2m + 1}{2}\Big\rfloor \\
& = & \Big\lfloor\frac{2m}{2}  + \frac{1}{2}\Big\rfloor\\
& = & m + \Big\lfloor\frac{1}{2}\Big\rfloor \\
& = & m \\
& = & \Big\lceil\frac{2m}{2}\Big\rceil \\
& = & \Big\lceil\frac{n}{2}\Big\rceil
\end{eqnarray*}

\textbf{Case 2:} $n$ is odd ($2 \not|$  $n$, or $n = 2m + 1$ for some $m \in \mathbb{N}$) 
\begin{eqnarray*}
\Big\lfloor\frac{n+1}{2}\Big\rfloor & = & \Big\lfloor\frac{(2m + 1) +1}{2}\Big\rfloor \\
& = & \Big\lfloor\frac{2(m + 1)}{2}\Big\rfloor \\
& = & m + 1 \\
& = & m + \Big\lceil\frac{1}{2}\Big\rceil \\
& = & \frac{2m}{2} + \Big\lceil\frac{1}{2}\Big\rceil \\
& = & \Big\lceil\frac{2m}{2} + \frac{1}{2}\Big\rceil \\
& = & \Big\lceil\frac{2m + 1}{2}\Big\rceil  \\
& = & \Big\lceil\frac{n}{2}\Big\rceil
\end{eqnarray*}

Thus, since a number $n \in \mathbb{N}$ can only be even or odd, we can conclude that for any $n \in \mathbb{N}, \Big\lfloor\frac{n+1}{2}\Big\rfloor = \Big\lceil\frac{n}{2}\Big\rceil$.
\end{sol}

\begin{prob}{1-b}
Prove that for any $n \in \mathbb{N}, \lfloor n/2\rfloor + 1 = \lceil (n+1)/2 \rceil$.
\end{prob}
\begin{sol}
\textbf{Case 1:} $n$ is even ($2$ $|$ $n$, or $n = 2m$ for some $m \in \mathbb{N}$) 
\begin{eqnarray*}
\Big\lfloor\frac{n}{2} \Big\rfloor + 1 & = & \Big\lfloor\frac{2m}{2} \Big\rfloor + 1\\
& = & \frac{2m}{2} + 1 \\
& = & \frac{2m}{2} + \Big\lceil\frac{1}{2}\Big\rceil \\
& = & \Big\lceil\frac{2m}{2}\Big\rceil + \Big\lceil\frac{1}{2}\Big\rceil \\
& = & \Big\lceil\frac{2m}{2} + \frac{1}{2}\Big\rceil \\
& = & \Big\lceil \frac{2m + 1}{2} \Big\rceil \\
& = & \Big\lceil\frac{n+1}{2} \Big\rceil
\end{eqnarray*}

\textbf{Case 2:} $n$ is odd ($2 \not|$  $n$, or $n = 2m + 1$ for some $m \in \mathbb{N}$) \\
\begin{eqnarray*}
\Big\lfloor\frac{n}{2}\Big\rfloor + 1 & = & \Big\lfloor\frac{2m+1}{2}\Big\rfloor + 1 \\
& = & \Big\lfloor\frac{2m}{2} + \frac{1}{2}\Big\rfloor + 1 \\
& = & m + \Big\lfloor\frac{1}{2}\Big\rfloor + 1 \\
& = & m + 1 \\
& = & \frac{2(m+1)}{2} \\
& = & \Big\lceil\frac{2(m+1)}{2}\Big\rceil \\
& = & \Big\lceil\frac{2m + 2}{2}\Big\rceil \\
& = & \Big\lceil\frac{(2m + 1) + 1}{2}\Big\rceil \\
& = & \Big\lceil\frac{n+1}{2}\Big\rceil
\end{eqnarray*}

Thus, since a number $n \in \mathbb{N}$ can only be even or odd, we can conclude that for any $n \in \mathbb{N}, \Big\lfloor\frac{n}{2}\Big\rfloor + 1 = \Big\lceil\frac{n+1}{2}\Big\rceil$.
\end{sol}

\begin{prob}{1-c}
Let $D(n) = T(n+1) - T(n)$. Determine a recurrence for $D(n)$.
\end{prob}
\begin{sol}

Let $D(n) = T(n+1) - T(n)$. If we let $n = 1$ be the base case for the recurrence as in $T(n)$, we obtain the following:
\begin{eqnarray*}
D(1) & = & T(2) - T(1) \\
& = &  T\Big(\Big\lceil\frac{2}{2}\Big\rceil\Big) + T\Big(\Big\lfloor\frac{2}{2}\Big\rfloor\Big) + 2 - 0 \\
& = & T(1) + T(1) + 2 \\
& = & 2
\end{eqnarray*}
Thus, we can see that $D(1) = 2$. We now seek the general case for $D(n)$ by expanding the its representation using the definition for $T(n)$, as shown below.
\begin{eqnarray*}
D(n) & = & T(n+1) - T(n) \\
& = & \Bigg(T\Big(\Big\lceil\frac{n+1}{2}\Big\rceil\Big) + T\Big(\Big\lfloor\frac{n+1}{2}\Big\rfloor\Big) + (n+1)\Bigg) - \Bigg(T\Big(\Big\lceil\frac{n}{2}\Big\rceil\Big) + T\Big(\Big\lfloor\frac{n}{2}\Big\rfloor\Big) + n\Bigg) \\
& = & T\Big(\Big\lceil\frac{n+1}{2}\Big\rceil\Big) + T\Big(\Big\lfloor\frac{n+1}{2}\Big\rfloor\Big) - T\Big(\Big\lceil\frac{n}{2}\Big\rceil\Big) - T\Big(\Big\lfloor\frac{n}{2}\Big\rfloor\Big) + 1 \\
& = & T\Big(\Big\lceil\frac{n+1}{2}\Big\rceil\Big) + T\Big(\Big\lfloor\frac{n}{2}\Big\rfloor\Big) + 1 \\
& = & T\Big(\Big\lfloor\frac{n}{2}\Big\rfloor + 1\Big) + T\Big(\Big\lfloor\frac{n}{2}\Big\rfloor\Big) + 1
\end{eqnarray*}
Now, by observing that this expression takes the same form as $D(n)$, we obtain the following:
\begin{eqnarray*}
D(n) & = & T\Big(\Big\lfloor\frac{n}{2}\Big\rfloor + 1\Big) + T\Big(\Big\lfloor\frac{n}{2}\Big\rfloor\Big) + 1 \\
& = & D(\Big\lfloor\frac{n}{2}\Big\rfloor) + 1
\end{eqnarray*}
Now, putting these results together, we obtain the following recurrence for $D(n)$:
\begin{eqnarray*}
D(1) & = & 2 \\
D(n) & = & D(\Big\lfloor\frac{n}{2}\Big\rfloor) + 1
\end{eqnarray*}
\end{sol}

\begin{prob}{1-d}
Prove using the strong form of induction that for any $n \in \mathbb{N}$, if $n \geq 1$ then $D(n) = \lfloor \tlg(n) \rfloor + 2$.
\end{prob}
\begin{sol}
\textbf{Base ($n = 1$)} \\
By the definition of $D(n)$, we know the following:
\begin{eqnarray*}
D(1) & = & 2 \\ 
& = & 0 + 2 \\
& = & \tlg(1) + 2 \\
& = & \Big\lfloor\tlg(1)\Big\rfloor + 2 
\end{eqnarray*}

\textbf{Induction ($n > 1$)} \\
Assume that $D(k) = \Big\lfloor\tlg(k)\Big\rfloor + 2$ for some $k \in \mathbb{N}$ such that $2 \leq k < n$. We now show that $D(n) = \Big\lfloor\tlg(n)\Big\rfloor + 2$.
\begin{eqnarray*}
D(n) & = &  D(\Big\lfloor\frac{n}{2}\Big\rfloor) + 1 \\
& = & \Big(\Big\lfloor\tlg(\Big\lfloor\frac{n}{2}\Big\rfloor)\Big\rfloor + 2\Big) + 1 \\
& = & \Big(\Big\lfloor\tlg(\Big\lfloor\frac{n}{2}\Big\rfloor)\Big\rfloor + 1\Big) + 2 \\
& = & \Big(\Big\lfloor\tlg(\Big\lfloor\frac{n}{2}\Big\rfloor)\Big\rfloor + \tlg(2)\Big) + 2 \\
& = & \Big(\Big\lfloor\tlg(\Big\lfloor\frac{n}{2}\Big\rfloor) + \tlg(2)\Big\rfloor\Big) + 2 \\
& = & \Big(\Big\lfloor\tlg(2\Big\lfloor\frac{n}{2}\Big\rfloor)\Big\rfloor\Big) + 2 \\
& = & \Big(\Big\lfloor\tlg(n)\Big\rfloor\Big) + 2 \\
& = & \Big\lfloor\tlg(n)\Big\rfloor + 2
%
%& = & \Big(\Big\lfloor\tlg(n) - \tlg(2)\Big\rfloor + 2\Big) + 1 \\
%& = & \Big(\Big\lfloor\tlg(n) - 1\Big\rfloor + 2\Big) + 1 \\
%& = & \Big((\Big\lfloor\tlg(n)\Big\rfloor - 1) + 2\Big) + 1 \\
%& = & \Big(\Big\lfloor\tlg(n)\Big\rfloor + 1\Big) + 1 \\
%& = & \Big\lfloor\tlg(n)\Big\rfloor + 2
\end{eqnarray*}
Thus, $D(n) = \Big\lfloor\tlg(n)\Big\rfloor + 2$, as desired.
\end{sol}

\begin{prob}{1-e}
Then prove that $T(n) - T(1) = \sum_{k=1}^{n-1} D(k)$, and show that an immediate consequence is that $T(n) = \sum_{k=1}^{n-1}\lfloor \tlg(k) \rfloor + 2$.
\end{prob}
\begin{sol}
By the definition of $D(n)$, we observe the following:
\begin{eqnarray*}
\sum_{k=1}^{n-1}D(k) & = & \sum_{k=1}^{n-1}\Big(T(k + 1) - T(k)\Big) \\
& = & \big(T(2) - T(1)\big) + \big(T(3) - T(2)\big) + \big(T(4) - T(3)\big) + ... + \big(T(n) - T(n-1)\big) \\
& = & T(n) - T(1) 
\end{eqnarray*}
Therefore, since $\sum_{k=1}^{n-1}D(k)$ turns into a telescoping sum, we see that it collapses to $T(n) - T(1)$, and since $D(n) = \Big\lfloor\tlg(n)\Big\rfloor + 2$, we also know the following:
\begin{eqnarray*}
T(n) - T(1) & = & \sum_{k=1}^{n-1}D(k) \\
& = & \sum_{k=1}^{n-1}\big(\Big\lfloor\tlg(k)\Big\rfloor + 2\big)
\end{eqnarray*}
Thus, we see that $T(n) - T(1) = \sum_{k=1}^{n-1}\big(\lfloor\tlg(k)\rfloor + 2\big)$, and since $T(1) = 0$, we conclude that $T(n) = \sum_{k=1}^{n-1}\big(\lfloor\tlg(k)\rfloor + 2\big)$.
\end{sol}

\begin{prob}{1-f}
Now show that $T(n) = \sum_{k=1}^{n-1}\lfloor \tlg(k) \rfloor + 2$ implies that $T(n) = O(n\tlog(n))$.
\end{prob}
\begin{sol}
Using the fact that $T(n) = \sum_{k=1}^{n-1}\big(\Big\lfloor\tlg(k)\Big\rfloor + 2\big)$. We now evaluate this summation as follows:
\begin{eqnarray*}
T(n) & = & \sum_{k=1}^{n-1}\big(\lfloor\tlg(n)\rfloor + 2\big) \\ 
& = & \sum_{k=1}^{n-1}\lfloor\tlg(k)\rfloor + \sum_{k=1}^{n-1}2 \\
& < & \sum_{k=1}^{n-1}\lfloor\tlg(n)\rfloor + \sum_{k=1}^{n-1}2 \\
& = & (n-1)\tlg(n) + 2(n-1) \\
& = & n\tlg(n) - \tlg(n) + 2n - 1 \\
& = & O(n\tlg(n)) \\
& = & O(n\tlog(n))
\end{eqnarray*}
\end{sol}

\begin{prob}{Extra credit}
Prove that for any $n \in \mathbb{N}, \lfloor\tlg(2\lfloor n/2 \rfloor) \rfloor = \lfloor \tlg (n) \rfloor$.
\end{prob}
\begin{sol}

\textbf{Case 1:} $n$ is even ($2$ $|$ $n$, or $n = 2m$ for some $m \in \mathbb{N}$)
\begin{eqnarray*}
\lfloor \tlg (2\lfloor n/2 \rfloor) \rfloor & = & \lfloor \tlg (2\lfloor 2m/2 \rfloor) \rfloor \\
& = & \lfloor \tlg (2\lfloor m \rfloor) \rfloor \\
& = & \lfloor \tlg (2m) \rfloor \\
& = & \lfloor \tlg (n) \rfloor
\end{eqnarray*}
\textbf{Case 2:} $n$ is odd ($2 \not| $  $n$, or $n = 2m + 1$ for some $m \in \mathbb{N}$)
\begin{eqnarray*}
\lfloor \tlg (2\lfloor n/2 \rfloor) \rfloor & = & \lfloor \tlg (2\lfloor (2m + 1)/2 \rfloor) \rfloor \\
& = & \lfloor \tlg (2\lfloor (2m/2) + (1/2) \rfloor) \rfloor \\
& = & \lfloor \tlg (2\lfloor m + (1/2) \rfloor) \rfloor \\
& = & \lfloor \tlg (2(m + \lfloor 1/2 \rfloor) \rfloor \\
& = & \lfloor \tlg (2m + 2\lfloor 1/2 \rfloor ) \rfloor \\
& = & \lfloor \tlg (2m + 1) \rfloor \\
& = & \lfloor \tlg (n) \rfloor
\end{eqnarray*}
\end{sol}

\end{document}
