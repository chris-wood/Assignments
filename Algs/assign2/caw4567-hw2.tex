\documentclass[11pt]{article}

\usepackage{thumbpdf, amssymb, amsmath, amsthm, microtype,
	    graphicx, verbatim, listings, color, fancybox}
\usepackage[pdftex]{hyperref}
%\usepackage[margin=1in]{geometry}
\usepackage{cawsty}
\usepackage{fullpage}
\usepackage{pseudocode}
\usepackage{verbatim}

\newcommand{\tlg}{\text{ lg}}
\newcommand{\tln}{\text{ ln}}

%\setlength{\parindent}{0pt}

\linespread{1.2}

\begin{document}
\arltitle{4005-800 Algorithms}{Homework 2}
\begin{prob}{1}
\end{prob}
\begin{sol} 

The time complexity of the recurrence $F_{n}$ can be defined as follows.

\begin{eqnarray*}
T_{F}(0) & = & 1 \\
T_{F}(1) & = & 1 \\
T_{F}(n) & = & T_{F}(n-1) + T_{F}(n-2)
\end{eqnarray*}

The solution to $T_{F}(n)$ can be solved using the method of homogeneous equations, which yields the result that $T_{F}(n) = \Theta(\phi^{n})$, where $\phi = \frac{1 + \sqrt{5}}{2}$ (the golden ratio).

\end{sol}

\begin{prob}{2}
\end{prob}
\begin{sol}

The time complexity of the $fibIt$ routine can be found by solving the recurrence relation that defines $fibIt$. More specifically, we can such a recurrence relation for $fibIt$ by analyzing the number of additions performed, which corresponds to the following equation.

\begin{eqnarray*}
T_{f}(0) & = & 0 \\
T_{f}(1) & = & 0 \\
T_{f}(n) & = & T_{f}(n-1) + 1
\end{eqnarray*}

This is because there is only one addition made in each recursive call from $f(n;a,b)$ to $f(n-1;b,a+b)$, and there are no additions made in the two cases where $n = 0$ and $n = 1$.

In order to solve this recurrence relation we can expand out the expression and attempt to identify the pattern. This process is shown below.

\begin{eqnarray*}
T_{f}(n) & = & T_{f}(n - 1) + 1 \\
& = & (T_{f}(n - 2) + 1) + 1 = T_{f}(n - 2) + 2 \\
& = & (T_{f}(n - 3) + 1) + 2 = T_{f}(n - 3) + 3 \\
& = & ... \\
& = & (T_{f}(n - k) + 1) + k = T_{f}(n - k) + k
\end{eqnarray*}

Based on this pattern, we can reach the first base case of this recurrence relation ($T_{f}(1)$) when $(n - k) = 1$, meaning that $k = (n - 1)$. Thus, we have the following.

\begin{eqnarray*}
T_{f}(n) & = & T_{f}(n - (n - 1)) + (n - 1)  \\
& = & T_{f}(1) + (n - 1) \\
& = & 0 + (n - 1) \\
& = & n - 1
\end{eqnarray*}

Based on this observation we can see that $T_{f}(n) \in \Theta(n)$, or simply $T_{f}(n) = \Theta(n)$. 

%However, to verify this, we perform the substitution method on this recurrence relation. To start, we also %assume that $T_{f}(n - 1) \in \Theta(n)$. Then, we perform the induction as follows.

%\begin{eqnarray*}
%T_{f}(n) & = & T_{f}(n - 1) + 1 \\
%& = & \Theta(n) + 1 \\
%& = & n + (n - 1) \\
%& = & n - 1
%\end{eqnarray*}

%%TODO: is it necessary to perform substitution to solve the rest of this?

\end{sol}

\begin{prob}{3}
\end{prob}
\begin{sol}

\textbf{Base ($n = 0$)} \\
When $n = 0$, we have the following equality.

\begin{eqnarray*}
L_{0}(a,b) & = & (f(0;a,b),f(1;a,b)) \\
& = & (a, b)
\end{eqnarray*}

\textbf{Induction ($n > 0$)} \\
First, we assume that $L^{n}(a,b) = (f(n;a,b),f(n + 1;a,b))$. Now we show that $L^{n+1}(a,b) = (f(n + 1;a,b),f(n + 2;a,b))$.

\begin{eqnarray*}
L^{n + 1}(a,b) & = & L(L^{n}(a,b)) \text{ (by multiplication powers)}\\
& = & L(f(n;a,b),f(n + 1;a,b)) \text{ (by induction)} \\
& = & (f(n + 1;a,b), f(n;a,b) + f(n + 1;a,b)) \text{ (by definition of $L$)} \\
& = & (f(n + 1;a,b), f(n + 2;a,b)) \text{ (by Theorem 1)}
\end{eqnarray*}

Thus, $L^{n+1}(a,b) = (f(n + 1;a,b),f(n + 2;a,b))$, as desired. Therefore, we know that $f(n;a,b) = (L^{n}(a,b))_{1}$.

\end{sol}

\begin{prob}{4}
\end{prob}
\begin{sol}
TODO: talk about representation, mention multiplication algorithm for logn time, give fibpow listing, list time complexity \\
TODO: matrix \\
TODO: repeated squaring \\
TODO: code here \\
TODO: time complexity
\end{sol}

\begin{prob}{5-a}
Write down the definition of pseudo-polynomial time.
\end{prob}
\begin{sol}
\begin{define}
Pseudo-polynomial time is the complexity class that encompasses all functions $f(n)$ that run in polynomial time in the numeric value of $n$, which is exponential in the length of $n$. 
\end{define}
\end{sol}

\begin{prob}{5-b}
Is $fib$ a pseudo-polynomial time algorithm? Explain.
\end{prob}
\begin{sol}
\begin{define}
TODO
\end{define}
\end{sol}

\begin{prob}{5-c}
Is $fibIt$ a pseudo-polynomial time algorithm? Explain.
\end{prob}
\begin{sol}
\begin{define}
TODO
\end{define}
\end{sol}

\begin{prob}{5-d}
Is $fibPow$ a pseudo-polynomial time algorithm? Explain.
\end{prob}
\begin{sol}
\begin{define}
TOOD
\end{define}
\end{sol}

\end{document}
