\documentclass[11pt]{article}

\usepackage{thumbpdf, amssymb, amsmath, amsthm, microtype,
	    graphicx, verbatim, listings, color, fancybox}
\usepackage[pdftex]{hyperref}
%\usepackage[margin=1in]{geometry}
\usepackage{cawsty}
\usepackage{fullpage}
\usepackage{pseudocode}
\usepackage{verbatim}
\usepackage{multicol}

\usepackage{fancybox}
\usepackage{tikz}

\newcommand{\tlg}{\text{lg}}
\newcommand{\tln}{\text{ln}}
\newcommand{\tlog}{\text{log}}

\usepackage{algorithm}
%\usepackage{algorithmic}
\usepackage{amsmath}
\usepackage{amsthm}
\usepackage{algpseudocode}
\usepackage{algorithmicx}% http://ctan.org/pkg/algorithmicx
\usepackage{lipsum}% http://ctan.org/pkg/lipsum
\usepackage{xifthen}% http://ctan.org/pkg/xifthen
\usepackage{needspace}% http://ctan.org/pkg/needspace
\usepackage{hyperref}% http://ctan.org/pkg/hyperref

\usepackage{tikz}
\usetikzlibrary{arrows,%
                shapes,positioning}

\tikzstyle{vertex}=[circle,fill=black!25,minimum size=20pt,inner sep=0pt]
\tikzstyle{selected vertex} = [vertex, fill=red!24]
\tikzstyle{edge} = [draw,thick,-]
\tikzstyle{weight} = [font=\small]
\tikzstyle{selected edge} = [draw,line width=5pt,-,red!50]
\tikzstyle{ignored edge} = [draw,line width=5pt,-,black!20]

\allowdisplaybreaks[1]

% ================ ALGORITHM ENVIRONMENT ================
\newcounter{numberedAlg}% Algorithm counter
\newenvironment{numberedAlg}[1][]%
  {% \begin{numberedAlg}[#1]
    \needspace{2\baselineskip}% At least 2\baselineskip required, otherwise break
    \noindent \rule{\linewidth}{1pt} \endgraf% Top rule
    \refstepcounter{numberedAlg}% For correct reference of algorithm
    \centering \textsc{Algorithm}~\thenumberedAlg%
    \ifthenelse{\isempty{#1}}{}{:\ #1}% Typeset name (if provided)
  }{% \end{numberedAlg}
  \noindent \rule{\linewidth}{1pt}% Bottom rule
  }%

%\setlength{\parindent}{0pt}

\linespread{1.2}

\begin{document}
\cawtitle{4005-800 Algorithms}{Homework 7}

\begin{prob}{1 - 34.2-1}
Consider the language GRAPH-ISOMORPHISM = $\{\langle G_1, G_2 \rangle : G_1$ and $G_2$ are isomorphic graphs $\}$. Prove that GRAPH-ISOMORPHISM $\in NP$ by describing a polynomial-time algorithm to verify the language.
\end{prob}
\begin{sol}

\end{sol}

\begin{prob}{2 - 34.2-10}
Prove that if $NP \not= co$-$NP$, then $P \not= NP$.
\end{prob}
\begin{sol}

\end{sol}

\begin{prob}{3 - 34.3-1}
Verify that the circuit in Figure $34.8(b)$ is unsatisfiable.
\end{prob}
\begin{sol}

\end{sol}

\begin{prob}{4 - 34.4-5}
Show that the problem of determining the satisfiability of boolean formulas in disjunctive normal form is polynomial-time solvable.
\end{prob}
\begin{sol}

\end{sol}

\begin{prob}{5 - 34.5-5}
The \textbf{set-partition problem} takes as input a set $S$ of numbers. The question is whether the numbers can be partitioned into two sets $A$ and $A' = S - A$ such that $\sum_{x \in A}x = \sum_{x \in A'} x$. Show that the set-partition problem is $NP$-complete.
\end{prob}
\begin{sol}

\end{sol}

\end{document}
