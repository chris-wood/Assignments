\documentclass[11pt]{article}

\usepackage{thumbpdf, amssymb, amsmath, amsthm, microtype,
	    graphicx, verbatim, listings, color, fancybox}
\usepackage[pdftex]{hyperref}
%\usepackage[margin=1in]{geometry}
\usepackage{cawsty}
\usepackage{fullpage}
\usepackage{pseudocode}
\usepackage{verbatim}

\newcommand{\tlg}{\text{ lg}}
\newcommand{\tln}{\text{ ln}}

%\setlength{\parindent}{0pt}

\linespread{1.2}

\begin{document}
\arltitle{4005-800 Algorithms}{Homework 1}
\begin{prob}{1}
\end{prob}
\begin{sol} 

The following table classifies the provided functions into equivalence class based on their order of growth in descending order.  \\

\begin{tabular}{|l|l|}
	\hline
	\textbf{Class} & \textbf{Functions} \\
	\hline 
	1 & $2^{2^{n+1}}$ \\
	2 & $2^{2^{n}}$ \\
	3 & $(n+1)!$ \\
	4 & $n!$ \\
	5 & $e^{n}$ \\
	6 & $n * 2^{n}$ \\
	7 & $2^{n} $ \\
	8 & $(\frac{3}{2})^{n}$ \\ 
	9 & $n^{\tlg \tlg n}$, $(\tlg n)^{\tlg n}$ \\
	10 & $(\tlg n)!$ \\
	11 & $n^{3}$ \\
	12 & $n^{2}$, $4^{\tlg n}$ \\
	13 & $\tlg(n!)$, $n \tlg n$ \\
	14 & $n$, $2^{\tlg n}$ \\
	15 & $(\sqrt[]{2})^{\tlg n}$ \\
	16 & $2^{\sqrt[]{2\tlg n}}$ \\
	17 & $\tlg^{2}n$ \\
	18 & $\tln n$ \\
	19 & $\sqrt[]{\tlg n}$ \\
	20 & $\tln \tln n$ \\
	21 & $2^{\tlg^{*}n}$ \\
	22 & $\tlg^{*}(\tlg n)$, $\tlg^{*}n$ \\
	23 & $\tlg(\tlg^{*}n)$ \\
	24 & $n^{\frac{1}{\tlg n}}$, $1$ \\
	\hline
\end{tabular} \\ \\
\end{sol}

\begin{prob}{2-a}
Using the definition of $O$, prove that $n = O(n^{2})$.
\end{prob}
\begin{sol} 
If $n \geq 1$, then $n^{2} \geq n$. Further, $0^{2} \geq 0$. Therefore, $n^{2} \geq n$ for any $n \in \mathbb{N}$. Thus, $cn^{2} \geq n$ when $n \geq 0$ and $c \geq 1$. By definition, this means that $n \in O(n^{2})$, or simply $n = O(n^{2})$.
\end{sol}

\begin{prob}{2-b}
Using the definition of $O$, prove that $n^{k} = O(n^{k'})$ if $k \leq k'$.
\end{prob}
\begin{sol} 

If $k \leq k'$, then it follows that $n^{k} \leq n^{k'}$ for any $n \in \mathbb{N}$. Thus, for constants $c \geq 1$, it is true that $n^{k} \leq n^{k'} \leq cn^{k'}$, or simply $n^{k} \leq cn^{k'}$, for all $n \geq 0$. Therefore, by definition, this means that $n^{k} \in O(n^{k'})$, or simply $n^{k} = O(n^{k'})$.

\end{sol}

\begin{prob}{3}
Write a function $fib$ that implements the recurrence relation for the Fibonacci numbers. What is the smallest $n$ such that you notice $fib$ running slowly?
\end{prob}
\begin{sol} 
The source code for the $fib$ routine (written in Python) is listed below. It is also included in the electronic submission.

\begin{lstlisting}
def fib(n):
	if (n == 0):
		return 0
	elif (n == 1):
		return 1
	else:
		return (fib(n - 1) + fib(n - 2))
\end{lstlisting}

The smalest value of $n$ that starts to yields long execution times is $n = 31$.

%TODO: explain the time complexity of this guy by solving with second order nonlinear homogeneous equations!
\end{sol}

\begin{prob}{4-a}
Prove using the strong form of mathematical induction that for any $n \in \mathbb{N}$ if $n > 1$ then $f(n;a,b)$ = $f(n-1;a,b) + f(n-2;a,b)$.
\end{prob}
\begin{sol} 
Based on the defintion for $f$ and the problem constraints, we need only consider $n = 2$ as the base case for induction since it is the first valid value of $n$ for which $f$ is true. \\

\textbf{Base ($n = 2$)} \\
By definition, we know that $f(2;a,b) = f(1;b, a + b) = (a+b)$. Further, by definition we know that $f(1;a,b)+ f(0;a,b) = b + a = (a + b)$. Thus, $f(2;a,b) = f(1;a,b)+ f(0;a,b)$, as desired.

\textbf{Induction ($n > 2$)} \\
Assume that $f(k;a,b) = f(k-1;a,b)+ f(k-2;a,b)$ for some $k \in \mathbb{N}$ such that $3 \leq k < n$. We now show that $f(n;a,b) = f(n - 1;a,b)+ f(n - 2;a,b)$. This result is described below.

\begin{eqnarray*}
f(n;a,b) & = & f(n-1;b,a+b) \text{ (by definition)}\\
& = & f(n-2;b, a+b) + f(n-3;b,a+b) \text{ (by induction)}\\ 
& = & f(n-1;a,b) + f(n-2;a,b) \text{ (by definition of $f$)}
\end{eqnarray*}

Thus, $f(n;a,b) = f(n - 1;a,b)+ f(n - 2;a,b)$, as desired.

\end{sol}

\begin{prob}{4-b}
Prove using the strong form of mathematical induction that for any $n \in \mathbb{N}$, $F_{n} = f(n;0,1)$.
\end{prob}
\begin{sol} 
Based on the results from the previous problem, $f(n)$ can depend on both $f(n-1)$ and $f(n-2)$. Therefore, we have two base cases to cover, as shown below. \\

\textbf{Base ($n = 0$)} \\
By definition, $F_{0} = 0$, and $f(0;0,1) = 0$. Thus, $F_{0} = f(0;0,1)$.

\textbf{Base ($n = 1$)} \\
By definition, $F_{1} = 1$, and $f(1;0,1) = 1$. Thus, $F_{1} = f(1;0,1)$.

\textbf{Induction ($n > 1$)} \\
Assume that $F_{k} = f(k;0,1)$ for some $k \in \mathbb{N}$ such that $2 \leq k < n$. We now show that $F_{n} = f(n;0,1)$. This result is described below.

\begin{eqnarray*}
F_{n} & = & F_{n - 1} + F_{n - 2} \text{ (by definition)} \\
& = & f(n - 1;0,1) + f(n-2;0,1) \text{ (by induction)} \\
& = & f(n;0,1) \text{ (by definition from problem 4-a)}
\end{eqnarray*}

Thus, $F_{n} = f(n;0,1)$, as desired.

\end{sol}

\begin{prob}{5}
Write a function $fibIt$ that implements $f$. Does $fibIt$ also run slowly on the value of $n$ that you found made $fib$ run slowly?
\end{prob}
\begin{sol} 
The source code for the $fibIt$ routine (written in Python) is listed below. It is also included in the electronic submission.

\begin{lstlisting}
def fibIt(n, a, b):
	if (n == 0):
		return a
	elif (n == 1):
		return b
	else:
		return fibIt(n - 1, b, a + b)
\end{lstlisting}
\end{sol}

Based on empirical observations, $fibIt$ \textbf{does not} run slowly on the same value of $n$ that made $fib$ run slowly. In fact, $fibIt$ will execute in a reasonable amount of time up to the point where the size of the recursive call stack is too large for the system to maintain in memory. 

%TODO: explain why (why oh why is this the case? - put it in a complexity class?)

\end{document}
